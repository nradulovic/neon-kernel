This is a bare-\/kernel, the result of two weeks of work. It is intended for a bigger project which already includes some synchronization mechanisms. The initial idea was that this kernel would only provide minimal functionality for context switching, but the kernel implementation went so nicely that I think that it would be great to share it.

e\-Solid is a collection of resources for embedded system design and this Real-\/\-Time kernel is only a piece of that collection. Because of that fact remember\-: {\itshape there are (still) no synchronization or I\-P\-C mechanisms in this kernel}, and it can be viewed as a preemptive Round-\/\-Robin scheduler, only.

\subsubsection*{T\-O\-D\-O list}


\begin{DoxyItemize}
\item Integrate a profiling system (memory/stack usage, cpu usage...)
\item test, test, test...
\end{DoxyItemize}

\subsection*{Using e\-Solid -\/ Real-\/\-Time Kernel}

\subsubsection*{Configuration and ports}

Configuration is done in two files\-: {\ttfamily \hyperlink{kernel__cfg_8h}{kernel\-\_\-cfg.\-h}} (port independent settings) and in {\ttfamily cpu\-\_\-cfg.\-h} (port depended settings, located in port structure). Currently, kernel is ported only to A\-R\-Mv7-\/\-M architecture range of microcontrollers. It was tested on S\-T\-M32\-F100 series of microcontrollers, but it should work, with minimal modifications, on any A\-R\-Mv7-\/\-M C\-P\-U. Some other ports like A\-V\-R-\/\-G\-C\-C are planned, too.

\subsubsection*{Building}

The kernel was built using arm-\/none-\/eabi G\-C\-C v4.\-7.\-3 compiler toolchain (from \href{https://launchpad.net/gcc-arm-embedded/+download}{\tt https\-://launchpad.\-net/gcc-\/arm-\/embedded/+download}) and binary was downloaded to the M\-C\-U using {\itshape texane} gdb-\/server. There are no makefiles, it is assumed that I\-D\-E will generate them for you.

\subparagraph*{Example for S\-T\-M32\-F10x family port}

There are two groups of source files which need to be compiled for A\-R\-Mv7-\/\-M architecture\-:
\begin{DoxyItemize}
\item \hyperlink{kernel_8c}{kernel.\-c}, semaphore.\-c, dbg.\-c in {\ttfamily ./src} source directory and
\item cpu.\-c in {\ttfamily ./port/arm-\/none-\/eabi-\/gcc/v7-\/m} port directory.
\end{DoxyItemize}

The following include paths are needed\-:
\begin{DoxyItemize}
\item {\ttfamily ./inc}
\item {\ttfamily ./port/arm-\/none-\/eabi-\/gcc/common}
\item {\ttfamily ./port/arm-\/none-\/eabi-\/gcc/v7-\/m}
\item {\ttfamily ./port/arm-\/none-\/eabi-\/gcc/stm32f10x}
\end{DoxyItemize}

\subsubsection*{Documentation}

Some documentation is available under Wiki \href{https://github.com/nradulovic/esolid-kernel/wiki}{\tt https\-://github.\-com/nradulovic/esolid-\/kernel/wiki}. Doxygen configuration and full documentation source files are available in {\ttfamily /doc} directory. Go to the directory {\ttfamily doc} create a directory named {\ttfamily kernel} and than run doxygen\-: \begin{DoxyVerb}# doxygen doxyfile-kernel
# doxygen doxyfile-kernel-port
\end{DoxyVerb}


This will generate H\-T\-M\-L, La\-Tex and man documentation in {\ttfamily ./doc/kernel} and {\ttfamily ./doc/kernel-\/port} directories, respectively.

\subsubsection*{Running}

To successfully use and run kernel you will need to study the kernel documentation. The documentation is still being written and some examples will be added later. 